\documentclass{article}\usepackage[]{graphicx}\usepackage[]{color}
%% maxwidth is the original width if it is less than linewidth
%% otherwise use linewidth (to make sure the graphics do not exceed the margin)
\makeatletter
\def\maxwidth{ %
  \ifdim\Gin@nat@width>\linewidth
    \linewidth
  \else
    \Gin@nat@width
  \fi
}
\makeatother

\definecolor{fgcolor}{rgb}{0.345, 0.345, 0.345}
\newcommand{\hlnum}[1]{\textcolor[rgb]{0.686,0.059,0.569}{#1}}%
\newcommand{\hlstr}[1]{\textcolor[rgb]{0.192,0.494,0.8}{#1}}%
\newcommand{\hlcom}[1]{\textcolor[rgb]{0.678,0.584,0.686}{\textit{#1}}}%
\newcommand{\hlopt}[1]{\textcolor[rgb]{0,0,0}{#1}}%
\newcommand{\hlstd}[1]{\textcolor[rgb]{0.345,0.345,0.345}{#1}}%
\newcommand{\hlkwa}[1]{\textcolor[rgb]{0.161,0.373,0.58}{\textbf{#1}}}%
\newcommand{\hlkwb}[1]{\textcolor[rgb]{0.69,0.353,0.396}{#1}}%
\newcommand{\hlkwc}[1]{\textcolor[rgb]{0.333,0.667,0.333}{#1}}%
\newcommand{\hlkwd}[1]{\textcolor[rgb]{0.737,0.353,0.396}{\textbf{#1}}}%

\usepackage{framed}
\makeatletter
\newenvironment{kframe}{%
 \def\at@end@of@kframe{}%
 \ifinner\ifhmode%
  \def\at@end@of@kframe{\end{minipage}}%
  \begin{minipage}{\columnwidth}%
 \fi\fi%
 \def\FrameCommand##1{\hskip\@totalleftmargin \hskip-\fboxsep
 \colorbox{shadecolor}{##1}\hskip-\fboxsep
     % There is no \\@totalrightmargin, so:
     \hskip-\linewidth \hskip-\@totalleftmargin \hskip\columnwidth}%
 \MakeFramed {\advance\hsize-\width
   \@totalleftmargin\z@ \linewidth\hsize
   \@setminipage}}%
 {\par\unskip\endMakeFramed%
 \at@end@of@kframe}
\makeatother

\definecolor{shadecolor}{rgb}{.97, .97, .97}
\definecolor{messagecolor}{rgb}{0, 0, 0}
\definecolor{warningcolor}{rgb}{1, 0, 1}
\definecolor{errorcolor}{rgb}{1, 0, 0}
\newenvironment{knitrout}{}{} % an empty environment to be redefined in TeX

\usepackage{alltt}
\usepackage{graphicx}

% Set page margins
\usepackage{geometry}
\geometry{verbose,tmargin=1in,bmargin=1in,lmargin=1in,rmargin=1in}
% Remove paragraph indenting
\setlength\parindent{0pt}
% Add hyperlinks
\usepackage{hyperref}

\title{STAT 243: Problem Set 5}
\author{Jin Rou New [jrnew]}
\date{\today}



\IfFileExists{upquote.sty}{\usepackage{upquote}}{}
\begin{document}
\maketitle

\section{Problem 1}
There was an overflow, since the result is smaller than the smallest number that can be represented, i.e. $e^{-1022}$. They should instead calculate the log likelihood by taking the sum of the logarithms of the probabilities.

\begin{knitrout}
\definecolor{shadecolor}{rgb}{0.969, 0.969, 0.969}\color{fgcolor}\begin{kframe}
\begin{alltt}
\hlkwd{options}\hlstd{(}\hlkwc{digits} \hlstd{=} \hlnum{10}\hlstd{)}
\hlkwd{load}\hlstd{(}\hlstr{"ps5prob1.Rda"}\hlstd{)}
\hlstd{p} \hlkwb{<-} \hlkwd{c}\hlstd{(}\hlnum{1}\hlopt{/}\hlstd{(}\hlnum{1} \hlopt{+} \hlkwd{exp}\hlstd{(}\hlopt{-}\hlstd{X} \hlopt \hlstd{beta)))}
\hlstd{likelihood} \hlkwb{<-} \hlkwd{prod}\hlstd{(}\hlkwd{dbinom}\hlstd{(y, n, p))} \hlcom{# Value of 0 obtained}
\hlstd{likelihood}
\end{alltt}
\begin{verbatim}
## [1] 0
\end{verbatim}
\begin{alltt}
\hlstd{log_likelihood} \hlkwb{<-} \hlkwd{sum}\hlstd{(}\hlkwd{log}\hlstd{(}\hlkwd{dbinom}\hlstd{(y, n, p)))}
\hlstd{log_likelihood}
\end{alltt}
\begin{verbatim}
## [1] -1862.33088
\end{verbatim}
\end{kframe}
\end{knitrout}

We can then calculate the likelihood by hand as either $e^{\ensuremath{-1862.3309}}$ or $10^{\ensuremath{-1862.3309}}/ln(10) = 10^{\ensuremath{-1862.3309}}/2.3026$.

%================================================================================
\section{Problem 2}
(a) We can expect 16 decimal places of accuracy at most for our result.

\begin{knitrout}
\definecolor{shadecolor}{rgb}{0.969, 0.969, 0.969}\color{fgcolor}\begin{kframe}
\begin{alltt}
\hlkwd{options}\hlstd{(}\hlkwc{digits} \hlstd{=} \hlnum{20}\hlstd{)}
\hlstd{num} \hlkwb{<-} \hlnum{1.000000000001}
\hlstd{num}
\end{alltt}
\begin{verbatim}
## [1] 1.0000000000010000889
\end{verbatim}
\end{kframe}
\end{knitrout}

(b), (c) and (d)
No, \texttt{sum()} does not give the right accuracy, but only up to 15 decimal places. In Python, it gives 0 decimal places whether I use the base \texttt{sum()} or from np. In both R and Python, using a \texttt{for} loop to do the summation with 1 as the first value in the vector do not give the right answer, but only a result with 0 decimal places of accuracy. With 1 as the last value in the vector, the right answer with 16 decimal places of accuracy is obtained.

\begin{knitrout}
\definecolor{shadecolor}{rgb}{0.969, 0.969, 0.969}\color{fgcolor}\begin{kframe}
\begin{alltt}
\hlcom{# In R}
\hlkwd{options}\hlstd{(}\hlkwc{digits} \hlstd{=} \hlnum{20}\hlstd{)}
\hlstd{x} \hlkwb{<-} \hlkwd{c}\hlstd{(}\hlnum{1}\hlstd{,} \hlkwd{rep}\hlstd{(}\hlnum{1e-16}\hlstd{,} \hlnum{10000}\hlstd{))}
\hlkwd{identical}\hlstd{(num,} \hlkwd{sum}\hlstd{(x))}
\end{alltt}
\begin{verbatim}
## [1] FALSE
\end{verbatim}
\begin{alltt}
\hlkwd{sum}\hlstd{(x)} \hlcom{# 15 dp of accuracy}
\end{alltt}
\begin{verbatim}
## [1] 1.0000000000009996448
\end{verbatim}
\begin{alltt}
\hlstd{sum_x} \hlkwb{<-} \hlnum{0}
\hlkwa{for} \hlstd{(i} \hlkwa{in} \hlnum{1}\hlopt{:}\hlstd{(}\hlkwd{length}\hlstd{(x)))}
  \hlstd{sum_x} \hlkwb{<-} \hlstd{sum_x} \hlopt{+} \hlstd{x[i]}
\hlkwd{identical}\hlstd{(num, sum_x)}
\end{alltt}
\begin{verbatim}
## [1] FALSE
\end{verbatim}
\begin{alltt}
\hlstd{sum_x} \hlcom{# 0 dp of accuracy}
\end{alltt}
\begin{verbatim}
## [1] 1
\end{verbatim}
\begin{alltt}
\hlstd{y} \hlkwb{<-} \hlkwd{c}\hlstd{(}\hlkwd{rep}\hlstd{(}\hlnum{1e-16}\hlstd{,} \hlnum{10000}\hlstd{),} \hlnum{1}\hlstd{)}
\hlstd{sum_y} \hlkwb{<-} \hlnum{0}
\hlkwa{for} \hlstd{(i} \hlkwa{in} \hlnum{1}\hlopt{:}\hlstd{(}\hlkwd{length}\hlstd{(y)))}
  \hlstd{sum_y} \hlkwb{<-} \hlstd{sum_y} \hlopt{+} \hlstd{y[i]}
\hlkwd{identical}\hlstd{(num, sum_y)}
\end{alltt}
\begin{verbatim}
## [1] TRUE
\end{verbatim}
\begin{alltt}
\hlstd{sum_y} \hlcom{# 16 dp of accuracy, right answer}
\end{alltt}
\begin{verbatim}
## [1] 1.0000000000010000889
\end{verbatim}
\end{kframe}
\end{knitrout}

\begin{knitrout}
\definecolor{shadecolor}{rgb}{0.969, 0.969, 0.969}\color{fgcolor}\begin{kframe}
\begin{alltt}
# In Python
import numpy as np
import decimal
num = 1.000000000001
x = np.array([1e-16]*(10001))
x[0] = 1
print(np.equal(num, sum(x)))
print(decimal.Decimal(sum(x))) # 0 dp of accuracy

print(np.equal(num, np.sum(x)))
print(decimal.Decimal(np.sum(x))) # 0 dp of accuracy

sum_x = 0
for x_select in x:
  sum_x += x_select
print(np.equal(num, sum_x))
print(decimal.Decimal(sum_x)) # 0 dp of accuracy

y = np.array([1e-16]*(10001))
y[len(y)-1] = 1
sum_y = 0
for y_select in y:
  sum_y += y_select
print(np.equal(num, sum_y))
print(decimal.Decimal(sum_y)) # 16 dp of accuracy, right answer
\end{alltt}

\begin{verbatim}
## False
## 1
## False
## 1
## False
## 1
## True
## 1.0000000000010000889005823410116136074066162109375
\end{verbatim}
\end{kframe}
\end{knitrout}

(e) It suggests that R's \texttt{sum()} function does not simply sum from left to right, but either sums the numbers of higher precision first before the 

(f) The \texttt{sum} function in R calls a sum function written in C. The C source code is available at: \url{https://github.com/wch/r-source/blob/trunk/src/main/summary.c}. In C, the \texttt{do\_summary} function calls \texttt{rsum} to carry out summation of doubles.

\begin{knitrout}
\definecolor{shadecolor}{rgb}{0.969, 0.969, 0.969}\color{fgcolor}\begin{kframe}
\begin{alltt}
\hlcom{# Code for sum()}
\hlstd{sum}
\end{alltt}
\begin{verbatim}
## function (..., na.rm = FALSE)  .Primitive("sum")
\end{verbatim}
\end{kframe}
\end{knitrout}
\end{document}

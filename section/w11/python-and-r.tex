\documentclass{article}

\usepackage{url}
%% Define a new 'leo' style for the package that will use a smaller font.
\makeatletter
\def\url@leostyle{%
  \@ifundefined{selectfont}{\def\UrlFont{\sf}}{\def\UrlFont{\small\ttfamily}}}
\makeatother
%% Now actually use the newly defined style.
\urlstyle{leo}


%% for inline R code: if the inline code is not correctly parsed, you will see a message
\newcommand{\rinline}[1]{SOMETHING WRONG WITH knitr}
%% begin.rcode setup, include=FALSE, cache=FALSE
% opts_chunk$set(fig.path='figure/latex-', cache.path='cache/latex-')
% read_chunk('scope.py')
% read_chunk('memory.py')
% read_chunk('randomwalk.R')
% read_chunk('randomwalk.py')
% read_chunk('float.py')
%% end.rcode

\title{Python inspired by problem set 4}
\date{November 3, 2014}
\author{Jarrod Millman\\ Statistics 243\\ UC Berkeley}

\begin{document}

\maketitle

In groups of two to three, I want you to spend about 10 minutes on each of the
following questions.  For most of the questions, you will need to use IPython
to try out the code snippets.  While you are discussing things, I will
circulate among the groups to answer questions and observe.  After working on
each question for 10 minutes in small groups, we will have a group discussion
for about 10 minutes.  And so on.

You may encounter aspects of Python that you haven't seen before.  Hopefully,
you will be able to get some sense of what the code snippet does by trying it
out with IPython.  If you can't get something working from IPython, please ask
me for clarification or help.

\begin{enumerate}

\item Scope

You can add an attribute to a function at any point during runtime.  As long as
the attribute is assigned a value prior to being needed, everything works.  If
you try to use an attribute before it is created (through assignment), Python
will raise an \texttt{AttributeError}.

%% begin.rcode  myfunc, engine='python'
%% end.rcode

If you call the function now, you should get the following error
message:\footnote{ Sorry about the curly quotes, \texttt{knitr}'s Python engine
doesn't properly handle single quotes.  You will notice that I use double
quotes in the Python snippets.  Normally, I would use single quotes in many
instances as it is less obtrusive and doesn't require the use of the
\texttt{Shift} key.}

%% begin.rcode  myfunc1, engine='python', eval=FALSE
%% end.rcode

That error means that \texttt{myfunc.table} has not been created yet.  To
create it just assign it a value.  In this case you will want to create
an empty dictionary.

%% begin.rcode  myfunc2, engine='python', eval=FALSE
%% end.rcode

Alternatively, you could initially populate it with some already known or
precalculated entries.  For example, you could save the resulting dictionary
from an early run of your code to disk.  Then when you want to use the
code again, you could load the saved dictionary from disk.\footnote{If you
need to save your dictionary to disk, you may want to use \texttt{JSON}.
\texttt{JSON} is an open standard, human-readable text file format that
is widely used as an alternative the \texttt{XML}. And is one of the most
popular methods for exchanging data on the web.}

Now use \texttt{type(myfunc)} and \texttt{type(myfunc.table)} to examine the
type of object that \texttt{myfunc} and \texttt{myfunc.table} are labels for.
Check that \texttt{myfunc.table} is still an empty dictionary.  Now call
\texttt{myfunc()} on an argument (e.g., 3) and look to see what happens to
\texttt{myfunc.table}.  Try calling \texttt{myfunc()} with different types of
arguments (e.g., give it a string or a list).  Do you get an error message?
Look at \texttt{myfunc.table}.  Did it change?  If so, what happened and why?

This is an example of how you could use Python to memoize a
function.\footnote{Python 3 provides a decorator to simplify this process: \\*
\url{https://docs.python.org/3/library/functools.html#functools.lru_cache}}
Memoization is a way to speed up a function by caching results once they've
been computed, so that you don't have to compute them again.  Can you think of
cases where this would be useful?

\item Memory

As I mentioned in an earlier section, variables are not their values in Python
(think ``I am not my name, I am the person named XXX").  Certain objects in
Python are mutable (e.g., lists, dictionaries), while other objects are
immutable (e.g., tuples, strings).  Many objects can be composite (e.g., a list
of dictionaries or a dictionary of lists, tuples, and strings).  For this
example, you are going to explore mutable and immutable objects by examining
how they compose.

First, create a list, a tuple, and a dictionary containing the list and the
tuple as values.

%% begin.rcode  mutable, engine='python', eval=FALSE
%% end.rcode

Look at the objects you've created.\footnote{Make sure you see which object is
composite.  If you have time, you may wish to examine whether you can further
nest objects.  Can you create a dictionary with a list of dictionaries of
lists?}  Check their types.  Use the \texttt{id} function to find the memory
addresses of the various objects.  For instance, you might try something like
this:

%% begin.rcode  mutable1, engine='python', eval=FALSE
%% end.rcode

What happens to \texttt{l1} when you modify \texttt{d["a"][1]}?  What about
when you modify \texttt{d["a"]}?  You might try:

%% begin.rcode  mutable2, engine='python', eval=FALSE
%% end.rcode

Similarly, what happens when you modify \texttt{d["b"][1]}? What about when you
modify \texttt{d["b"]}?  You might try:

%% begin.rcode  mutable3, engine='python', eval=FALSE
%% end.rcode

Now try to make a copy of \texttt{d}.  First, create a new variable by assign
the old variable to it.

%% begin.rcode  mutable4, engine='python', eval=FALSE
%% end.rcode

What happened?  If you use tab-completion on \texttt{d.} from IPython, you will
see it has a \texttt{copy} method. 

%% begin.rcode  mutable4, engine='python', eval=FALSE
%% end.rcode

What does that do?  Finally make a ``deepcopy" of \texttt{d}.

%% begin.rcode  mutable5, engine='python', eval=FALSE
%% end.rcode

Can you explain the differences between the various ways the above copies work?

\clearpage 
\item Random walks

For question 4, you were asked to implement a random walk function. As a
reminder, here is the example solution.

%% begin.rcode  q4.sol
%% end.rcode

Here is an implementation in Python.

%% begin.rcode  p1, engine='python'
%% end.rcode

Compare the two versions.  Try to find parallels in the implementation (I tried
to mimic the R implementation mostly so hopefully they look somewhat similar to
you).  Do you prefer one over the other?  How about specific code snippets.

Note the syntax used to create the variable \texttt{walk} in the definition of
the \texttt{random\_2d\_walk()} function.  The argument to the
\texttt{np.array()} constructor is created using list comprehension.  This
notation should remind you of the mathematical set-builder
notation.\footnote{Creating sets using the set-builder notation is known as set
comprehension.}

In the Python implementation, I am not checking the type and value of my input.
If I wanted to so, I could something like this:

%% begin.rcode  p1.a, engine='python', eval=FALSE
%% end.rcode

It is generally not consider Pythonic to check argument types explicitly.  The
focus is on whether the interface is implemented.  This is called \emph{duck
typing}.  It is normally considered desirable to let the function error if
the input is invalid.  It you want to handle the error more directly, then a
common approach is to try using the input and then handle any exceptions
explicitly.

\clearpage 
Here is the example solution for creating an S3 class in R. 
%% begin.rcode  q4class.sol
%% end.rcode

\clearpage 
Here is an object-oriented implementation in Python.\footnote{In order to make
the \texttt{position} method act like an attribute, you could use the
\texttt{@property} decorator.}
%% begin.rcode  p2, engine='python'
%% end.rcode

\begin{figure}[ht]
  \centering
  \includegraphics[width=.8\textwidth]{../test.png}
  \caption{Plot of a random walk on a 2-dimensional grid.}
  \label{fig:figure1}
\end{figure}

Since \texttt{knitr} is not able to display figures generated by Python, I've
included the figure generated by the above code in Figure~\ref{fig:figure1}.
I've also used the \texttt{print} statement to include the output.  If you are
working at an interactive Python prompt, you can omit it.  You also might
prefer to use \texttt{plt.show()} rather than saving the figure to disk with
\texttt{plt.savefig()}.

Look through this implementation and see if you can understand how it works.
Note the use of \texttt{self}.\footnote{The use of \texttt{self} is a
convention.  You can use another name (e.g., \texttt{this}), but it is probably
best to use the convention.}  It is the first argument to the class methods and
is how the methods are able to refer to the attributes of an instance.

Compare the two class implementations.  Are there any major differences in how
classes are written in Python and R?  Is one of the implementations clearer or
easy to read?  Is that due to your familiarity with R?  Are you able to work
out what Python is doing?  Does its syntax seem obscure?   Type the class
definition at the IPython prompt and create an instance of it.  Use tab-completion
to explore the object.  Are you able to figure out how to access and work with
its attributes and methods?

After considering the similarities and differences in how Python and R
implement object-oriented programming, can you think of any design criteria
that might explain the differences between how R and Python implement
object-oriented programming?  For instance, would one be easier to use
on-the-fly while working interactively?  Does one implementation seem
like it would be better to use in a larger codebase?  Why or why not?

\item Bonus

For problem set 5, you examined underflow when adding many small values to a
large value.  Normally you would use the \texttt{sum()} method of the
\texttt{ndarray} rather than Python's builtin \texttt{sum()} function.  If you
wanted to use higher precision, then you would specify it explicitly.

%% begin.rcode  float128, engine='python'
%% end.rcode

Note that \texttt{math.fsum} uses an alternative algorithm to ensure that it
doesn't loose precision.  Look at the doctring for \texttt{np.sum}.  Can you
see why specifying the \texttt{dtype} works?

You may find the following code and discussion interesting.\footnote{
\url{http://code.activestate.com/recipes/393090/}}
Finally, for the interested, Python also supports rational arthimetic.\footnote{
\url{https://docs.python.org/2/library/fractions.html}}

For a more details on floating-point arthimetic, please see the classic text by
David
Goldberg.\footnote{\url{http://docs.oracle.com/cd/E19957-01/806-3568/ncg_goldberg.html}}

\end{enumerate}

\end{document}


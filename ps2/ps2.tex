\documentclass{article}\usepackage[]{graphicx}\usepackage[]{color}
%% maxwidth is the original width if it is less than linewidth
%% otherwise use linewidth (to make sure the graphics do not exceed the margin)
\makeatletter
\def\maxwidth{ %
  \ifdim\Gin@nat@width>\linewidth
    \linewidth
  \else
    \Gin@nat@width
  \fi
}
\makeatother

\definecolor{fgcolor}{rgb}{0.345, 0.345, 0.345}
\newcommand{\hlnum}[1]{\textcolor[rgb]{0.686,0.059,0.569}{#1}}%
\newcommand{\hlstr}[1]{\textcolor[rgb]{0.192,0.494,0.8}{#1}}%
\newcommand{\hlcom}[1]{\textcolor[rgb]{0.678,0.584,0.686}{\textit{#1}}}%
\newcommand{\hlopt}[1]{\textcolor[rgb]{0,0,0}{#1}}%
\newcommand{\hlstd}[1]{\textcolor[rgb]{0.345,0.345,0.345}{#1}}%
\newcommand{\hlkwa}[1]{\textcolor[rgb]{0.161,0.373,0.58}{\textbf{#1}}}%
\newcommand{\hlkwb}[1]{\textcolor[rgb]{0.69,0.353,0.396}{#1}}%
\newcommand{\hlkwc}[1]{\textcolor[rgb]{0.333,0.667,0.333}{#1}}%
\newcommand{\hlkwd}[1]{\textcolor[rgb]{0.737,0.353,0.396}{\textbf{#1}}}%

\usepackage{framed}
\makeatletter
\newenvironment{kframe}{%
 \def\at@end@of@kframe{}%
 \ifinner\ifhmode%
  \def\at@end@of@kframe{\end{minipage}}%
  \begin{minipage}{\columnwidth}%
 \fi\fi%
 \def\FrameCommand##1{\hskip\@totalleftmargin \hskip-\fboxsep
 \colorbox{shadecolor}{##1}\hskip-\fboxsep
     % There is no \\@totalrightmargin, so:
     \hskip-\linewidth \hskip-\@totalleftmargin \hskip\columnwidth}%
 \MakeFramed {\advance\hsize-\width
   \@totalleftmargin\z@ \linewidth\hsize
   \@setminipage}}%
 {\par\unskip\endMakeFramed%
 \at@end@of@kframe}
\makeatother

\definecolor{shadecolor}{rgb}{.97, .97, .97}
\definecolor{messagecolor}{rgb}{0, 0, 0}
\definecolor{warningcolor}{rgb}{1, 0, 1}
\definecolor{errorcolor}{rgb}{1, 0, 0}
\newenvironment{knitrout}{}{} % an empty environment to be redefined in TeX

\usepackage{alltt}
\usepackage{graphicx}

% Set page margins
\usepackage{geometry}
\geometry{verbose,tmargin=1in,bmargin=1in,lmargin=1in,rmargin=1in}
% Remove paragraph indenting
\setlength\parindent{0pt}

\title{STAT 243: Problem Set 2}
\author{Jin Rou New}
\date{\today}



\IfFileExists{upquote.sty}{\usepackage{upquote}}{}
\begin{document}
\maketitle

\section{Problem 1}
First open an input file connection to the \texttt{bz2} file using \texttt{bzfile}, then read in the first block of data with \texttt{read.csv}. Subset the data according to the given column index and subset value, then stratify the data with \texttt{split} and output stratified data to respective \texttt{bz2} files by opening output file connections using again \texttt{bzfile} and writing to the connections with \texttt{read.table}. Close output file connections, then read in the next block of data. While this reading step does not produce an error, repeat previous data processing steps.

% # Read and process blocks while there are still rows to be read and processed
% while (!class(data) == "try-error") {
%   data_subset <- data[as.character(data[, var_subset]) == value_subset, ] # Subset data
%   data_strata_list <- split(data_subset, data_subset[, var_stratify]) # Stratify data
%   for (stratum in names(data_strata_list)) { # Output data strata to respective bz2 files
%     data_filepath_out <- file.path(output_dir, paste0(stratum, ".csv.bz2"))
%     con_out <- bzfile(data_filepath_out, "at") # Open output file connection (for appending)
%     write.table(data_strata_list[[stratum]], row.names = FALSE, col.names = FALSE, sep = ",", 
%                 file = con_out, append = TRUE) # Write to file connection
%     close(con_out) # Close output file connection
%   }
%   data <- try(read.csv(con, header = FALSE, nrows = nrows_block)) # Read next block
% }

\begin{knitrout}
\definecolor{shadecolor}{rgb}{0.969, 0.969, 0.969}\color{fgcolor}\begin{kframe}
\begin{alltt}
 \hlcom{#!/usr/bin/Rscript}

 \hlcom{# Parse arguments}
 \hlstd{args} \hlkwb{<-} \hlkwd{commandArgs}\hlstd{(}\hlnum{TRUE}\hlstd{)}
 \hlkwd{stopifnot}\hlstd{(}\hlkwd{length}\hlstd{(args)} \hlopt{==} \hlnum{4}\hlstd{)}
 \hlstd{data_filepath} \hlkwb{<-} \hlstd{args[}\hlnum{1}\hlstd{]}
 \hlstd{var_stratify} \hlkwb{<-} \hlkwd{as.integer}\hlstd{(args[}\hlnum{2}\hlstd{])} \hlcom{# Column index to stratify on}
 \hlstd{var_subset} \hlkwb{<-} \hlkwd{as.integer}\hlstd{(args[}\hlnum{3}\hlstd{])} \hlcom{# Column index to subset on}
 \hlstd{value_subset} \hlkwb{<-} \hlkwd{as.character}\hlstd{(args[}\hlnum{4}\hlstd{])} \hlcom{# Value to subset on}

 \hlstd{output_dir} \hlkwb{<-} \hlstr{"output"}
 \hlkwd{dir.create}\hlstd{(output_dir,} \hlkwc{showWarnings} \hlstd{=} \hlnum{FALSE}\hlstd{)} \hlcom{# Set up output directory}
 \hlstd{con} \hlkwb{<-} \hlkwd{bzfile}\hlstd{(data_filepath,} \hlstr{"rt"}\hlstd{)} \hlcom{# Open input file connection (for reading in text mode)}
 \hlstd{nrows_block} \hlkwb{<-} \hlnum{50000} \hlcom{# Read from input file in blocks of size nrows_block}
 \hlstd{header} \hlkwb{<-} \hlkwd{read.csv}\hlstd{(con,} \hlkwc{header} \hlstd{=} \hlnum{FALSE}\hlstd{,} \hlkwc{nrows} \hlstd{=} \hlnum{1}\hlstd{)}
 \hlstd{data} \hlkwb{<-} \hlkwd{read.csv}\hlstd{(con,} \hlkwc{header} \hlstd{=} \hlnum{FALSE}\hlstd{,} \hlkwc{nrows} \hlstd{= nrows_block)} \hlcom{# Read in first block}
 \hlcom{# Read and process blocks while there are still rows to be read and processed}
 \hlkwa{while} \hlstd{(}\hlopt{!}\hlkwd{class}\hlstd{(data)} \hlopt{==} \hlstr{"try-error"}\hlstd{) \{}
   \hlstd{data_subset} \hlkwb{<-} \hlstd{data[}\hlkwd{as.character}\hlstd{(data[, var_subset])} \hlopt{==} \hlstd{value_subset, ]} \hlcom{# Subset data}
   \hlstd{data_strata_list} \hlkwb{<-} \hlkwd{split}\hlstd{(data_subset, data_subset[, var_stratify])} \hlcom{# Stratify data}
   \hlkwa{for} \hlstd{(stratum} \hlkwa{in} \hlkwd{names}\hlstd{(data_strata_list)) \{} \hlcom{# Output data strata to respective bz2 files}
     \hlstd{data_filepath_out} \hlkwb{<-} \hlkwd{file.path}\hlstd{(output_dir,} \hlkwd{paste0}\hlstd{(stratum,} \hlstr{".csv.bz2"}\hlstd{))}
     \hlstd{con_out} \hlkwb{<-} \hlkwd{bzfile}\hlstd{(data_filepath_out,} \hlstr{"at"}\hlstd{)} \hlcom{# Open output file connection (for appending)}
     \hlkwd{write.table}\hlstd{(data_strata_list[[stratum]],} \hlkwc{row.names} \hlstd{=} \hlnum{FALSE}\hlstd{,} \hlkwc{col.names} \hlstd{=} \hlnum{FALSE}\hlstd{,} \hlkwc{sep} \hlstd{=} \hlstr{","}\hlstd{,}
                 \hlkwc{file} \hlstd{= con_out,} \hlkwc{append} \hlstd{=} \hlnum{TRUE}\hlstd{)} \hlcom{# Write to file connection}
     \hlkwd{close}\hlstd{(con_out)} \hlcom{# Close output file connection}
   \hlstd{\}}
   \hlstd{data} \hlkwb{<-} \hlkwd{try}\hlstd{(}\hlkwd{read.csv}\hlstd{(con,} \hlkwc{header} \hlstd{=} \hlnum{FALSE}\hlstd{,} \hlkwc{nrows} \hlstd{= nrows_block))} \hlcom{# Read next block}
 \hlstd{\}}
 \hlkwd{closeAllConnections}\hlstd{()}
\end{alltt}
\end{kframe}
\end{knitrout}

\begin{knitrout}
\definecolor{shadecolor}{rgb}{0.969, 0.969, 0.969}\color{fgcolor}\begin{kframe}
\begin{alltt}
 chmod ugo+x subset-and-stratify.R
 data_filepath="data/AirlineData2006-2008.csv.bz2"
 var_stratify=1 # Year
 var_subset=18 # Flight destination
 value_subset="SFO"
 ./subset-and-stratify.R $data_filepath $var_stratify $var_subset $value_subset
\end{alltt}
\end{kframe}
\end{knitrout}
%======================================================================
\section{Problem 2}
(a)
\texttt{myFuns} is a list of 3 functions, each of which returns the value of \texttt{i}. Since \texttt{i} is not found in the function environment, \texttt{i} in the global environment is used. At the first evaluation, the value of \texttt{i} is 3 from the last run of the \texttt{for} loop. Hence, that value is returned and printed out thrice.

(b)
Again, \texttt{i} is being found in the global environment and its value is 1 at the first iteration of the \texttt{for} loop, 2 at the second iteration and so on.

(c)
\texttt{i} is now being found in the environment of the function \texttt{funGenerator} during both the third and fourth evaluations. The value of \texttt{i} is 3 at the end of the evaluation of the \texttt{for} loop.
%======================================================================
\section{Problem 3}
Functions are called in the following order:
\begin{enumerate}
  \item \texttt{sapply} with frame number 1 and objects in the frame are \texttt{FUN}, \texttt{simplify}, \texttt{USE.NAMES}, \texttt{X}
  \item \texttt{lapply} with frame number 2 and objects in the frame are \texttt{FUN}, \texttt{X}
  \item \texttt{FUN} with frame number 3 and the only object in the frame is \texttt{x}
\end{enumerate}
\end{document}
